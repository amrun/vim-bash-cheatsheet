\documentclass[a4paper,10pt,landscape]{article}
\usepackage{multicol}
\usepackage{calc}
\usepackage{ifthen}
\usepackage[landscape]{geometry}
\usepackage{amsmath,amsthm,amsfonts,amssymb}
\usepackage{color,graphicx,overpic}
\usepackage{hyperref}
\usepackage[utf8]{inputenc}
\usepackage{wrapfig}
\usepackage{array}
\usepackage{pifont}

%define colors
\definecolor{light-grey}{rgb}{0.5,0.5,0.5}




\pdfinfo{
  /Title (example.pdf)
  /Creator (TeX)
  /Producer (pdfTeX 1.40.0)
  /Author (Seamus)
  /Subject (Example)
  /Keywords (pdflatex, latex,pdftex,tex)}

% This sets page margins to .5 inch if using letter paper, and to 1cm
% if using A4 paper. (This probably isn't strictly necessary.)
% If using another size paper, use default 1cm margins.
\ifthenelse{\lengthtest { \paperwidth = 11in}}
    { \geometry{top=.5in,left=.5in,right=.5in,bottom=.5in} }
    {\ifthenelse{ \lengthtest{ \paperwidth = 297mm}}
        {\geometry{top=1cm,left=1cm,right=1cm,bottom=1cm} }
        {\geometry{top=1cm,left=1cm,right=1cm,bottom=1cm} }
    }

% Turn off header and footer
\pagestyle{empty}

% Redefine section commands to use less space
\makeatletter
\renewcommand{\section}{\@startsection{section}{1}{0mm}%
                                {-1ex plus -.5ex minus -.2ex}%
                                {0.5ex plus .2ex}%x
                                {\normalfont\large\bfseries}}
\renewcommand{\subsection}{\@startsection{subsection}{2}{0mm}%
                                {-1explus -.5ex minus -.2ex}%
                                {0.5ex plus .2ex}%
                                {\normalfont\normalsize\bfseries}}
\renewcommand{\subsubsection}{\@startsection{subsubsection}{3}{0mm}%
                                {-1ex plus -.5ex minus -.2ex}%
                                {1ex plus .2ex}%
                                {\normalfont\small\bfseries}}
\makeatother

% Define BibTeX command
\def\BibTeX{{\rm B\kern-.05em{\sc i\kern-.025em b}\kern-.08em
    T\kern-.1667em\lower.7ex\hbox{E}\kern-.125emX}}

% Don't print section numbers
\setcounter{secnumdepth}{0}


\setlength{\parindent}{0pt}
\setlength{\parskip}{0pt plus 0.5ex}

%My Environments
\newtheorem{example}[section]{Example}
% -----------------------------------------------------------------------

\begin{document}

\begin{center}
     \Large{BASH / VI Cheatsheet}  \footnotesize{V1.1} \\
\rule{\linewidth}{0.25pt}\\
\end{center}


\raggedright
\begin{multicols}{2}
\footnotesize


% multicol parameters
% These lengths are set only within the two main columns
%\setlength{\columnseprule}{0.25pt}
\setlength{\premulticols}{1pt}
\setlength{\postmulticols}{1pt}
\setlength{\multicolsep}{1pt}
\setlength{\columnsep}{2pt}

\section{\underline{VI / VIM}}
\vspace{3mm}
Der VI hat verschiedene Modi. Um einen Modus zu verlassen wird Esc benutzt. Um zb. etwas einzugeben muss man in den Insert Mode wechseln.\\
\vspace{2mm}
ZAL -\textgreater Zwischenablage\\
curs. -\textgreater cursor\\
pos. -\textgreater position\\
\vspace{2mm}
\begin{tabular}{l p{8cm}}
{\bf i} & insert text an curs. pos.\\\hline
{\bf a} & append text nach curs. pos.\\\hline
{\bf Esc} & Modus verlassen (zb. insert)\\\hline
{\bf dd} &  Zeile löschen und in ZAL\\\hline
{\bf yy} &  Zeile in ZAL\\\hline
{\bf m x} &  Marker x (beliebiger Buchstabe) an momentaner pos.\\\hline
{\bf ' x} &  Marker x (beliebiger Buchstabe) benutzen\\\hline
{\bf :s/abc/def/} & abc mit def beim ersten Vorkommen in der momentanen Zeile ersetzen\\\hline
{\bf :\%s/abc/def/} & abc mit def beim ersten Vorkommen auf einer Zeile im ganzen file ersetzen\\\hline
{\bf :\%s/abc/def/g} & abc mit def im ganzen File ersetzen\\\hline
{\bf cw} & 'change word' löscht das ganze folgende Wort und geht in den Insert Mode\\\hline
{\bf :syntax on} & Aktiviert Syntaxhighlighting\\\hline
{\bf :set number} & Zeigt Linenumbers\\\hline
{\bf :set nonumber} & Versteckt Linenumbers\\\hline
{\bf :w} & Speichern\\\hline
{\bf :wq} & Speichern und schliessen\\\hline
{\bf :q!} & Schliessen OHNE speichern\\\hline
{\bf u} & Rückgängig machen\\\hline
{\bf shift u oder :redo} & Redo\\\hline
%{\bf blah} & blah\\
\end{tabular}


\section{\underline{BASH}}
\vspace{3mm}
\begin{tabular}{>{\raggedright}p{5cm}>{\raggedright}p{8cm}}
{\bf CTRL + R} & Suche in abgesetzten Befehlen/History (nochmal CTRL + R für den nächsten Treffer)\tabularnewline\hline
{\bf cd} & Verzeichnis wechseln (.. eine Ebene tiefer)\tabularnewline\hline
{\bf command1 \&\& command2} & command1 wird ausgeführt, command2 nur wenn command1 ok war\tabularnewline\hline
{\bf cp file1 file2} & copy (-r für Rekursives kopieren)\tabularnewline\hline
{\bf mv} & move (auch zum umbenennen)\tabularnewline\hline
{\bf grep -rin 'abc' ./} & Suche nach dem String abc in allen Files des momentanen Verzeichnisses (./) rekursiv (-r), case insensitiv (-i), mit Anzeige der Zeilennummer und dem File in dem der String gefunden wurde\tabularnewline\hline
{\bf find -name '*.php'} & Suche nach Files welche mit .php enden, rekursiv ab dem momentanen Verzeichnis\tabularnewline\hline
{\bf man command} & Hilfe/Informationen zum Befehl command. / zum durchsuchen, q zum verlassen\tabularnewline\hline
{\bf scp file.txt benutzer@server:\Pisymbol{psy}{191}\newline/pfad/zum/speichern/} & Kopiert das File file.txt auf den remote-Server in den angegebenen Pfad (geht auch in die andere Richtung)\tabularnewline\hline
{\bf ctrl-z} & Den momentanen Prozess/Job in den Hintergrund schieben (mit mehreren möglich).\tabularnewline\hline
{\bf jobs} & Laufende Jobs (zb. jene im Hintergrund) anzeigen.\tabularnewline\hline
{\bf fg x} & Holt den Job mit der Nummer x in den Vordergrund (falls nur einer im Hintergrund ist, kann jener auch einfach mit "fg" geholt werden).\tabularnewline\hline
%{\bf blah} & blah\tabularnewline\hline
\end{tabular}


\end{multicols}
\end{document}
